\documentclass{article}
\usepackage[a4paper, total={6in, 8in}]{geometry}
\usepackage{graphicx}
\graphicspath{ {./temp/} }

\usepackage[utf8]{inputenc}

\title{CMSC 6950 Final Project - pymagicc}
\author{Prudhvi Kommareddi}
\date{June 2021}

\begin{document}
\maketitle

\section{Introduction}
Pymagicc\cite{Gieseke2018} is a Python interface for the Fortran-based reduced-complexity climate carbon cycle model MAGICC (Meinshausen, Raper, and Wigley 2011). Aiming at broadening the user base of MAGICC1, Pymagicc provides a wrapper around the MAGICC binary, which runs on Windows and has been published under a Creative Commons Attribution. NonCommercial-ShareAlike 3.0 Unported License. Pymagicc itself is licensed under the GNU Affero General Public License v3.0.

\section{Tasks}
This project utilises the Pymagicc module to achieve the below computational tasks and visualizations.

\subsection{Task 1- Generate Greenhouse Gas Emissions}
In this task, we read data from RCP2.6, RCP4.5, RCP6, RCP,8.5 scenario files, convert the data in MAGICData format to a pandas DataFrame, and then build visualizations to show Carbon Dioxide and Methane gas emission projections for RCP2.6 and RCP4.5 scenarios. 
\begin{figure}[h]
    \includegraphics[width=\textwidth,height=\textheight,keepaspectratio]{RCP26_Emissions_CH4}
\end{figure}

\begin{figure}
    \includegraphics[width=\textwidth,height=\textheight,keepaspectratio]{RCP26_Emissions_CO2}
\end{figure}

\begin{figure}
    \includegraphics[width=\textwidth,height=\textheight,keepaspectratio]{RCP45_Emissions_CH4}
\end{figure}

\begin{figure}
    \includegraphics[width=\textwidth,height=\textheight,keepaspectratio]{RCP45_Emissions_CO2}
\end{figure}

\clearpage
\subsection{Task 2- Generate Radiative forcing plot}
In this task, we run the MAGICC model on RCP2.6, RCP4.5, RCP6, RCP,8.5 scenarios and visualize the Radiative Forcing projections for each of the given projections from 1765 to 2100.
\begin{figure}[h]
\includegraphics[width=\textwidth,height=\textheight,keepaspectratio]{radiative_forcing}
\end{figure}

\bibliographystyle{unsrt}
\bibliography{refs}

\end{document}
