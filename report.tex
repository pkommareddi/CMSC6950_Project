\documentclass{article}
% \usepackage[a4paper, total={6in, 8in}]{geometry}
\usepackage{float}
\usepackage{graphicx}
\graphicspath{ {./temp/} }

\usepackage[utf8]{inputenc}

\title{CMSC 6950 Final Project - pymagicc}
\author{Prudhvi Kommareddi}
\date{June 2021}

\begin{document}
\maketitle

\section{Introduction}
Pymagicc\cite{Gieseke2018} is an open-source Python wrapper for the Fortran-based MAGICC climate model.
It makes it simpler to use and run the MAGICC model Windows application. Pymagicc comes with several
built-in data structures to model the several emission pathways. MAGICCData is the core data structure 
used to model the Representative Concentration Pathways (RCP) and comes with Pandas DataFrame like
functionality and also has the ability to make line plots.

MAGICC, which stands for Model for the Assessment of Greenhouse Gas Induced Climate Change, is a
climate model widely used to assess future Greenhouse gas emissions in climate policy analyses.
It is most prominently used by the Intergovernmental Panel on Climate Change (IPCC) for crucial scientific
publications and by many Integrated Assessment Models.


\section{Tasks}
This project utilizes the Pymagicc module to achieve the below computational tasks and visualizations.
While MAGICCData object provides built-in functionality to make line plots, this project uses the
Matplotlib library to generate plots. The majority of the tasks involve wrangling the scenario's data
to generate the desired line charts.

\textbf{Note:} All of the visualizations have been recreated from the examples in the
original git repository. Also, the data available in this project is time-series data,
and hence there was little room for experimentation.

\subsection{Task 1- Generate Greenhouse Gas Emissions}
The computational task involves reading data from RCP2.6, RCP4.5, RCP6, RCP,8.5 scenario files
and converting the data in MAGICData format to a pandas DataFrame. Using the data previously saved, we
build visualizations to show Carbon Dioxide and Methane gas emission projections for RCP2.6 and RCP4.5 scenarios.

\begin{figure}[ht]
    \includegraphics[width=\textwidth,height=\textheight,keepaspectratio]{RCP26_Emissions_CH4}
    \caption{RCP26 CH4 Emissions Projections}
\end{figure}

\begin{figure}
    \includegraphics[width=\textwidth,height=\textheight,keepaspectratio]{RCP26_Emissions_CO2}
    \caption{RCP26 CO2 Emissions Projections}
\end{figure}

\begin{figure}
    \includegraphics[width=\textwidth,height=\textheight,keepaspectratio]{RCP45_Emissions_CH4}
    \caption{RCP45 CH4 Emissions Projections}
\end{figure}

\begin{figure}
    \includegraphics[width=\textwidth,height=\textheight,keepaspectratio]{RCP45_Emissions_CO2}
    \caption{RCP45 CO2 Emissions Projections}
\end{figure}

\clearpage
\subsection{Task 2- Generate plots on the MAGICC model data}
In this task, we run the MAGICC model on RCP2.6, RCP4.5, RCP6, RCP,8.5 scenarios and visualize the Radiative Forcing projections for each of the given projections from 1765 to 2100.
\begin{figure}[H]
\includegraphics[width=\textwidth,height=\textheight,keepaspectratio]{Radiative_Forcing}
\caption{Radiative Forcing Projections}
\end{figure}

\begin{figure}[H]
\includegraphics[width=\textwidth,height=\textheight,keepaspectratio]{Surface_Temperature}
\caption{Surface Temperature Projections}
\end{figure}
    

\bibliographystyle{unsrt}
\bibliography{refs}

\end{document}
